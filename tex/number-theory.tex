\documentclass[article,10pt,oneside]{memoir}
\input{preambles}
\title{greatest common divisor}
\author{Frank Tsai}
\date{\today}
% \thanks{}

\newcommand{\modulo}{\mathrel{\mathrm{mod}}}

\begin{document}
\maketitle

\begin{defn}\label{defn:gcd}
  The \emph{greatest common divisor} (gcd) of two integers $a$ and $b$, is an integer $k$ such that
  \begin{itemize}
  \item $k \mid a$
  \item $k \mid b$
  \item $\forall k' \in \dZ.\,(k' \mid a) \wedge (k' \mid b) \imp (k' \mid k)$
  \end{itemize}
\end{defn}

\begin{rmk}
  Note that \cref{defn:gcd} allows nonunique gcds.
  In fact, if $k$ is a gcd of $a$ and $b$, then so is $-k$.
  This can be remedied by choosing the nonnegative one.
  We write $\gcd(a, b)$ for the nonnegative integer satisfying the conditions in \cref{defn:gcd}.
\end{rmk}

\begin{rmk}
  Although gcds is defined for integers, its computation can be done using just nonnegative integers (c.f., Lemmas \ref{lem:symmetry} and \ref{lem:sign}).
\end{rmk}

\cref{lem:fundamental} is the backbone lemma that enables us to device a recursive algorithm to compute gcds.
\begin{lem}\label{lem:fundamental}
  For any integers $a, b$, and $c$, $\gcd(a - cb, b) = \gcd(a,b)$.
\end{lem}
\begin{proof}
  Let $k = \gcd(a - cb, b)$.
  It follows by definition that $k \mid b$.
  Since $a - cb = dk$ for some integer $d$, we can express $a$ as $a = dk + cb$.
  But since $b$ is divisible by $k$, $a$ is also divisible by $k$.
  
  It remains to show that, for any integer $k'$, if $k'$ divides both $a$ and $b$, then $k'$ also divides $k$.
  To this end, it suffices to show that $k' \mid (a - cb)$ and that $k' \mid b$.
  The latter follows immediately by assumption.
  As for the former, since $k'$ divides both $a$ and $b$ by assumption, it follows that $k'$ divides $a - cb$.
\end{proof}

\begin{lem}\label{lem:symmetry}
  For all integers $a$ and $b$, $\gcd(a, b) = \gcd(b, a)$.
\end{lem}
\begin{proof}
  Exercise.
\end{proof}

\begin{lem}\label{lem:sign}
  For all integers $a$ and $b$, $\gcd(-a, b) = \gcd(a, b)$.
\end{lem}
\begin{proof}
  Exercise.
\end{proof}

\begin{lem}\label{lem:base}
  For any integer $a$, $\gcd(0,a) = a$.
\end{lem}
\begin{proof}
  Exercise.
\end{proof}

If we set $c := 1$ in \cref{lem:fundamental}, then we get $\forall a, b \in \dZ.\,\gcd(a-b, b) = \gcd(a, b)$.
This allows us to calculate the gcd of two integers as follows.

\begin{eg}\label{eg:c-1}
  \begin{align}
    \gcd(321,123) &= \gcd(198,123)\\
                  &= \gcd(75,123) = \gcd(123,75)\\
                  &= \gcd(48,75) = \gcd(75,48)\\
                  &= \gcd(27,48) = \gcd(48,27)\\
                  &= \gcd(21,27) = \gcd(27,21)\\
                  &= \gcd(6,21) = \gcd(21,6)\\
                  &= \gcd(15,6)\\
                  &= \gcd(9,6)\\
                  &= \gcd(3,6) = \gcd(6,3)\\
                  &= \gcd(3,3)\\
                  &= \gcd(0,3) = 3
  \end{align}
\end{eg}

Instead of choosing a fixed $c$ in each step, Euclid's algorithm chooses a more clever $c$.
This choice is given by Euclid's division lemma.
\begin{lem}[Euclid's division lemma]
  Given two integers $a$ and $b$, with $b \neq 0$, there are unique integers $q$ and $r$ such that
  \begin{itemize}
  \item $a = bq + r$
  \item $0 \leq r < |b|$, where $|b|$ is the absolute value of $b$
  \end{itemize}
\end{lem}

Now, to compute $\gcd(a, b)$, we can choose $c := q$, where $q$ is the unique integer such that $a = bq + r$.
Then, $\gcd(a, b) = \gcd(a - qb, b)$.
Note that $a - qb = r$, so we obtain a formula in terms of the remainder: $\gcd(a, b) = \gcd(r, b)$.
Now, \cref{eg:c-1} can be computed as follows.

\begin{eg}
  \begin{align}
    \gcd(321, 123) &= \gcd(321 \modulo 123, 123)\\
                   &= \gcd(75, 123) = \gcd(123, 75) = \gcd(123 \modulo 75, 75)\\
                   &= \gcd(48, 75) = \gcd(75, 48) = \gcd(75 \modulo 48, 48)\\
                   &= \gcd(27, 48) = \gcd(48, 27) = \gcd(48 \modulo 27, 27)\\
                   &= \gcd(21, 27) = \gcd(27, 21) = \gcd(27 \modulo 21, 21)\\
                   &= \gcd(6, 21) = \gcd(21, 6) = \gcd(21 \modulo 6, 6)\\
                   &= \gcd(3, 6) = \gcd(6, 3) = \gcd(6 \modulo 3, 3)\\
                   &= \gcd(0, 3) = 3
  \end{align}
\end{eg}

\begin{thm}
  In each step of Euclid's algorithm computing $\gcd(a, b)$, both arguments can be expressed as a combination of $a$ and $b$.
\end{thm}
\begin{proof}
  Let $n$ be the number of steps taken and $a_{n}$ and $b_{n}$ be the first and the second argument at step $n$, respectively.
  We need to show that $a_{n} = u_{1}a + v_{1}b$ and $b_{n} = u_{2}a + v_{2}b$ for some integers $u_{1}, v_{1}, u_{2}$, and $v_{2}$.

  The base case is clear.
  \[
    \gcd(a, b) = \gcd(1a + 0b, 0a + 1b).
  \]
  
  In the induction step, let us assume without loss of generality that $b_{n} < a_{n}$.
  We can always swap them if it is not the case.
  By definition, $a_{n+1} = r$, where $r = a_{n} - b_{n}q$, and $b_{n+1} = b_{n}$.
  By the induction hypothesis, $b_{n} = u_{2}a + v_{2}b$, so clearly $b_{n+1}$ can be expressed as a combination of $a$ and $b$.
  As for $a_{n+1}$, we have $a_{n} = u_{1}a + v_{1}b$ by the induction hypothesis, so
  \[
    a_{n+1} = a_{n} - b_{n}q = (u_{1}a + v_{1}b) - (v_{2}a + v_{2}b)q = (u_{1} - v_{2}q)a + (v_{1} - v_{2}q)b
  \]
\end{proof}

\begin{cor}[B\'ezout's lemma]
  If $k$ is $\gcd(a, b)$, then $k = ua + vb$ for some integers $u$ and $v$.
\end{cor}

\bibliographystyle{alpha}
\bibliography{all}

\end{document}
