\documentclass{beamer}
\newif\ifbeamer\beamertrue
\input{decls}
\tikzcdset{diagrams={ampersand replacement=\&}}
\usetikzlibrary{shapes.geometric, arrows}

% \usetheme{Warsaw}
% \setbeamertemplate{headline}{}

\usetheme[height=0cm]{Rochester}
\useinnertheme{circles}
% \useoutertheme[compress,subsection=false]{miniframes}
\usecolortheme{rose}
\setbeamertemplate{navigation symbols}{}

\title{Logistics}
\author[Frank]{Frank Tsai\inst{1}}
\institute{\inst{1} (SUNY at Buffalo)}
\date{}

\AtBeginSection[]
{
  \begin{frame}<beamer>{Outline}
    \tableofcontents[currentsection]
  \end{frame}
}

\begin{document}

\begin{frame}
  \maketitle
\end{frame}

\begin{frame}{Piazza}
  \begin{itemize}
  \item Questions and discussions should be posted on Piazza
  \item Reserve email for university bureaucracy (exam accommodation, etc)
  \item Include ``[CS191]'' in the email title
  \end{itemize}
  \vfill
  Sign-up link: \href{https://piazza.com/buffalo/spring2024/cs191}{https://piazza.com/buffalo/spring2024/cs191}
\end{frame}

\begin{frame}{About}{Piazza: \href{https://piazza.com/buffalo/spring2024/cs191}{https://piazza.com/buffalo/spring2024/cs191}}
  \begin{itemize}
  \item[] \textbf{Goal:} develop mathematical maturity (think like a mathematician/computer scientist)
  \item<2->[] \textbf{Why?}
  \item<3-> \textbf{System designers:} designed system has desired properties (e.g., passwords don't leak)
  \item<4-> \textbf{Programming language designers:} designed programming language has desired properties (e.g., correct programs don't crash)
  \item<5-> \textbf{Software engineers:} develop principled approach to solving problems 
  \item<6-> Computer science is intimately linked to mathematics (you will start to appreciate this later)
  \end{itemize}
\end{frame}

\begin{frame}{Recitations}{Piazza: \href{https://piazza.com/buffalo/spring2024/cs191}{https://piazza.com/buffalo/spring2024/cs191}}
  \begin{itemize}
  \item Attendance is not required
  \item Recitations supplement lectures
    \begin{itemize}
    \item<2-> Coq tutorials
    \item<2-> Supplementary lectures
    \item<2-> Homework hints
    \end{itemize}
  \end{itemize}
\end{frame}

\begin{frame}{Installing Coq}{Piazza: \href{https://piazza.com/buffalo/spring2024/cs191}{https://piazza.com/buffalo/spring2024/cs191}}
  \begin{enumerate}
  \item Go to \href{https://github.com/coq/platform/releases/tag/2023.03.0}{https://github.com/coq/platform/releases/tag/2023.03.0}
  \item If your OS is
    \begin{itemize}
    \item[] \textbf{Windows:} choose ``Windows (64 bit) installer for Coq 8.17.1''\vspace{1pt}
    \item[] \textbf{Mac (Intel):} choose ``macOS (Intel) installer for Coq 8.17.1''\vspace{1pt}
    \item[] \textbf{Mac (M1/M2/Apple Silicon/ARM):} choose ``macOS (M1/M2/Apple Silicon/ARM) installer for Coq 8.17.1''\vspace{1pt}
    \item[] \textbf{Linux:} choose ``Linux (Snap) installer for Coq 8.17.1''
    \end{itemize}
  \item You are free to leave once you have a working installation of Coq
  \end{enumerate}
  \begin{block}{Mac users}
    You probably need to give the installer permission to run.
    You can find the option under System Preferences > Privacy \& Security
  \end{block}
\end{frame}

\end{document}
